\documentclass{report}
\renewcommand*\thesection{\arabic{section}}
\usepackage{amssymb}
\usepackage{amsmath}
\usepackage{amsthm}
\newtheorem{definition}{Definition}
\usepackage[linesnumbered, lined, noend]{algorithm2e}
\usepackage{todonotes}
%\usepackage[cm]{fullpage}
\usepackage{hyperref}
\hypersetup{
    colorlinks=true,       % false: boxed links; true: colored links
    linkcolor=black,       % color of internal links                   red
    citecolor=black,       % color of links to bibliography            green
    filecolor=black,       % color of file links                       magenta
    urlcolor=blue         % color of external links                   cyan
    }
\newcommand{\BigTheta}[1]{\ensuremath{\operatorname{\Theta}\bigl(#1\bigr)}}

\newcommand{\AND}{\wedge}
\newcommand{\OR}{\vee}
\newcommand{\stefan}[1]{\todo[inline,color=red]{#1}}
\newcommand{\mikkel}[1]{\todo[inline,color=yellow]{#1}}
\newcommand{\bruno}[1]{\todo[inline,color=blue]{#1}}
\newcommand{\mikael}[1]{\todo[inline,color=green]{#1}}

\usepackage{tikz-er2}

%Math database join symbols

\def\ojoin{\setbox0=\hbox{$\bowtie$}%
  \rule[-.02ex]{.25em}{.4pt}\llap{\rule[\ht0]{.25em}{.4pt}}}
\def\leftouterjoin{\mathbin{\ojoin\mkern-5.8mu\bowtie}}
\def\rightouterjoin{\mathbin{\bowtie\mkern-5.8mu\ojoin}}
\def\fullouterjoin{\mathbin{\ojoin\mkern-5.8mu\bowtie\mkern-5.8mu\ojoin}}
\newcommand{\HRule}{\rule{\linewidth}{0.5mm}}
\begin{document}

\begin{titlepage}
\centering
{\LARGE Advanced Algorithms - Self-Study 5}
\HRule \\[0.5cm]
Bruno Thalmann\\
			Mikael E. Christensen\\
			Mikkel S. Larsen\\
			Stefan M. G. Micheelsen\\
			Martin Rasmussen\\
			Kasper Lind Sørensen			
\end{titlepage}

\section*{CLRS 33-2}
\paragraph{a.}We consider the leftmost point in layer $i$ and layer $i+1$.
The points are denoted as $p_i = (x_i, y_i)$ and $p_{i +1} = (x_{i +1}, y_{i +1})$ respectively.
We wish to show that $y_i > y_{i +1}$.
We do so by showing that $y_i \leq y_{i +1}$ cannot be true.

We split into two cases:
\begin{itemize}
\item \textbf{Assume} that $x_i > x_{i+1}$.
\\This assumption means that $p_{i+1}$ is not dominated by $p_i$.
But as $p_i$ is the leftmost point in layer $i$, this would mean that $p_{i+1}$ is part of layer $i$.
Thus we have a contradiction and $x_i > x_{i+1}$ cannot be true for $y_i \leq y_{i+1}$.
\item \textbf{Assume} that $x_i \leq x_{i+1}$.
\\This assumption means that $p_i$ is dominated by $p_{i+1}$.
This gives a direct contradiction as $p_{i+1}$ dominates $p_i$.
This $x_i \leq x_{i+1}$ cannot be true for $y_i \leq y_{i+1}$.
\end{itemize}

\paragraph{b.}We divide into the two cases described by the exercise:
\begin{itemize}
\item $j \leq k$: We have that $y_j < y$, thus the new point cannot be part of layers $L_{j+1}, L_{j+2} \dots L_k$ as per the the proof above.
Since $j$ is the minimum index such that $y_j < y$ it must also be true that the new point cannot be part of layers $L_{j-1}, L_{j-2} \dots L_1$.
Thus the point must be part of layer $L_j$.
\item $j = k + 1$
\end{itemize}

\end{document}