\documentclass{report}
\renewcommand*\thesection{\arabic{section}}
\usepackage{amssymb}
\usepackage{amsmath}
\usepackage[linesnumbered, lined, noend]{algorithm2e}
\usepackage{todonotes}
%\usepackage[cm]{fullpage}
\usepackage{hyperref}
\hypersetup{
    colorlinks=true,       % false: boxed links; true: colored links
    linkcolor=black,       % color of internal links                   red
    citecolor=black,       % color of links to bibliography            green
    filecolor=black,       % color of file links                       magenta
    urlcolor=blue         % color of external links                   cyan
    }
\newcommand{\BigTheta}[1]{\ensuremath{\operatorname{\Theta}\bigl(#1\bigr)}}

\newcommand{\AND}{\wedge}
\newcommand{\OR}{\vee}
\newcommand{\stefan}[1]{\todo[inline,color=red]{#1}}
\newcommand{\mikkel}[1]{\todo[inline,color=yellow]{#1}}
\newcommand{\bruno}[1]{\todo[inline,color=blue]{#1}}
\newcommand{\mikael}[1]{\todo[inline,color=green]{#1}}

\usepackage{tikz-er2.sty}

\newcommand{\HRule}{\rule{\linewidth}{0.5mm}}
\begin{document}

\begin{titlepage}
\centering
{\LARGE Advanced Algorithms - Self-Study 1}
\HRule \\[0.5cm]
Bruno Thalmann\\
			Mikael E. Christensen\\
			Mikkel S. Larsen\\
			Stefan M. G. Micheelsen\\
			Martin Rasmussen
\end{titlepage}

\chapter*{Self-Study 1}

\section*{CLRS 16-2, part a}
We wish to minimize the average completion time of the tasks, defined as:
$$\frac{1}{n}\sum_{i=1}^{n}c_i$$
As $\frac{1}{n}$ is a constant, the problem is equivalent to minimizing:
$$\sum_{i=1}^{n}c_i$$

To do this we simply sort the tasks by their processing time $p_i$, in ascending order.
Below we prove that this ordering minimizes the average completion time.

\paragraph{Proof}
Let $A$ be a list of tasks ordered as above.
We will then show that swapping two tasks will result in a larger average running time.

Let $a_j$ and $a_{j+1}$, be two tasks in $A$.
Note that we have that $p_j \leq p_{j+1}$ due to the ordering of tasks.

Since we only swap two adjacent tasks in $A$ we know that all tasks $a_1, \cdots, a_{j-1}$ have the same completion time before and after the swap.
The same is true for all tasks $a_{j+2},\cdots,a_n$.

We now determine the completion time $c_j$ and $c_{j+1}$ for the two tasks before the swap:
$$c_j=p_1+p_2+\dots+p_{j-1}+p_j$$
$$c_{j+1}=p_1+p_2+\dots+p_{j-1}+p_j + p_{j+1}$$

We can also determine completion times after the swap:
$$c'_j=p_1+p_2+\dots+p_{j-1}+p_{j+1}$$
$$c'_{j+1}=p_1+p_2+\dots+p_{j-1}+p_{j+1} + p_j$$

We subtract the sum of completion times after the swap from the sum of completion times before the swap.
Since the only difference between the two is values of the swapped completion times, we simply look at the difference between these:
$$c_j + c_{j+1} - (c'_j + c'_{j+1})= p_j - p_{j + 1} \leq 0$$
This shows that the summed completion time after swapping the two tasks is greater than the summed completion time before swapping the two.
Thus we have proved that the ordering is optimal.

\section*{Exercise 2 (CLRS 15-1)}

\subsection*{Bruteforce algorithm:}

\begin{algorithm}[H]
GETMAX(v,t)
\\
\If{v=t}{return 0}
dist $= -\infty$\\
\ForEach{e in v.E}{d = GETMAX(e.end, t) + e.w \\ \If{d $\ge$ dist}{dist = d}}
return dist
\end{algorithm}
The complexity for this algorithm is $2^V$.

\subsection*{Dynamic Programming Algorithm design:}

\begin{itemize}
\item{Each step: Choose next edge to pursue}
\item{Sub-problems: Choose next edge going out of vertice obtained from previously pursued edge}
\item{Trivial sub-problems: Return distance 0 when current vertice is equal to $t$ (goal vertice)}
\item{Remember optimal choices: Table with vertices and their greatest accumulated weight}
\item{Construct solution from rememberings: Use parent pointers for vertices to combine vertices with highest accumulated weight into a path with longest weight}
\end{itemize}

\stefan{skriv initialisering}
\begin{algorithm}[H]
GETMAX(v,t)\\
\If{table[v] $> - \infty $}{return table[v]}
\ForEach{e in v.E}{d = GETMAX(e.end,t) + e.w\\
\If{d $ > $ table[v]}{table[v]=d}} 
return table[v]
\end{algorithm}

\end{document}