\documentclass{report}
\renewcommand*\thesection{\arabic{section}}
\usepackage{amssymb}
\usepackage{amsmath}
\usepackage{amsthm}
\newtheorem{definition}{Definition}
\usepackage[linesnumbered, lined, noend]{algorithm2e}
\usepackage{todonotes}
%\usepackage[cm]{fullpage}
\usepackage{hyperref}
\hypersetup{
    colorlinks=true,       % false: boxed links; true: colored links
    linkcolor=black,       % color of internal links                   red
    citecolor=black,       % color of links to bibliography            green
    filecolor=black,       % color of file links                       magenta
    urlcolor=blue         % color of external links                   cyan
    }
\newcommand{\BigTheta}[1]{\ensuremath{\operatorname{\Theta}\bigl(#1\bigr)}}

\newcommand{\AND}{\wedge}
\newcommand{\OR}{\vee}
\newcommand{\stefan}[1]{\todo[inline,color=red]{#1}}
\newcommand{\mikkel}[1]{\todo[inline,color=yellow]{#1}}
\newcommand{\bruno}[1]{\todo[inline,color=blue]{#1}}
\newcommand{\mikael}[1]{\todo[inline,color=green]{#1}}

\usepackage{tikz-er2}

%Math database join symbols

\def\ojoin{\setbox0=\hbox{$\bowtie$}%
  \rule[-.02ex]{.25em}{.4pt}\llap{\rule[\ht0]{.25em}{.4pt}}}
\def\leftouterjoin{\mathbin{\ojoin\mkern-5.8mu\bowtie}}
\def\rightouterjoin{\mathbin{\bowtie\mkern-5.8mu\ojoin}}
\def\fullouterjoin{\mathbin{\ojoin\mkern-5.8mu\bowtie\mkern-5.8mu\ojoin}}
\newcommand{\HRule}{\rule{\linewidth}{0.5mm}}
\begin{document}

\begin{titlepage}
\centering
{\LARGE Database System - Mini-project}
\HRule \\[0.5cm]
Bruno Thalmann\\
			Mikael E. Christensen\\
			Mikkel S. Larsen\\
			Stefan M. G. Micheelsen
\end{titlepage}

\chapter*{Self-Study 1}

\section*{CLRS 16-2, part a}
We wish to minimize the average completion time of the tasks, defined as:
$$\frac{1}{n}\sum_{i=1}^{n}c_i$$
As $\frac{1}{n}$ is a constant, the problem is equivalent to minimizing:
$$\sum_{i=1}^{n}c_i$$

To do this we simply sort the tasks by their processing time $p_i$, in ascending order.
Below we prove that this ordering minimizes the average completion time.

\paragraph{Proof}
Let $A$ be a list of tasks ordered as above.
We will then show that swapping two tasks will result in a larger average running time.

Let $a_j$ and $a_k$, where $j < k$ be two distinct tasks in $A$.
Note that we have that $p_j \leq p_k$ due to the ordering of tasks.

We now determine the completion time $c_i$ of the tasks.


$c_j=p_1+p_2+\dots+p_{j-1}+p_j + p_{j+1} + p_{j+2} + \dots + p_n$\\
$c_{j}=p_1+p_2+\dots+p_{j-1}+p_j + p_{j+1} + p_{j+2} + \dots + p_n$\\

\end{document}