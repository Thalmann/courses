\section{ER Modeling}
\includegraphics{libraryER.pdf}

\section{Banking System}
\begin{description}
\item[customer:]{\{[\underline{SSN}, name, address]\}}
\item[phone:]{\{[\underline{number}, provider, contract]\}}
\item[has:]{\{[\underline{SSN $\rightarrow$ customer, number $\rightarrow$ phone}]\}}
\item[account:]{\{[\underline{SSN $\rightarrow$ customer, accountNumber}, type, balance]\}}
\item[statement:]{\{[\underline{accountNumber $\rightarrow$ account, ID}, date]\}}
\end{description}

\section{Relational Algebra}

\subsection{~}

\begin{tabular}{lllll}

\begin{tabular}{|c|c|}
\hline
\multicolumn{2}{|c|}{$\pi_{species, zooID}(animals)$} \\ \hline
\textbf{species} & \textbf{zooID} \\ \hline
giraffe & 1 \\ \hline
giraffe & 2 \\ \hline
giraffe & 3 \\ \hline
ape & 1 \\ \hline
ape & 2 \\ \hline
owl & 2 \\ \hline
owl & 1 \\ \hline
\end{tabular}
&
$\div$
&
\begin{tabular}{|c|}
\hline
$\pi_{zooID}(\sigma_{country='Germany'}(zoos))$ \\ \hline
\textbf{zooID} \\ \hline
1 \\ \hline
\end{tabular}
&
$=$
&
\begin{tabular}{|l|}
\hline
\textbf{species} \\ \hline
giraffe \\ \hline
ape \\ \hline
owl \\ \hline
\end{tabular}

\end{tabular}

\section{Relational Calculus}

\section{Functional Dependencies}
