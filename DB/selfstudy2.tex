\section{Entity Relationship Diagram}

\includegraphics[width=\textwidth]{ER.pdf}

\section{Relations}
\begin{itemize}
\item User: \{[\underline{Email: string}, Username: string, Password: string]\}

\item Movie: \{[\underline{DateOfRelease: datetime, Title: string}, Language: string]\}

\item RefersTo: \{[ \underline{FromDateOfRelease: datetime, FromTitle: string,}\\ \underline{ToDateOfRelease: datetime, ToTitle: string, Type: string}]\}

	Foreign key \{ FromDateOfRelease, FromTitle \} $ \rightarrow $ \{Movie.DateOfRelease, Movie.Title\}
	
	Foreign key \{ ToDateOfRelease, ToTitle \} $ \rightarrow $ \{Movie.DateOfRelease, Movie.Title\}

\item Rates: \{[\underline{DateOfRelease: datetime, Title: string, Email $ \rightarrow $ User}, Value: integer]\}

Foreign key \{DOR, Title\}  $ \rightarrow $ \{Movie.DateOfRelease, Movie.Title\}

\item Genre: \{[\underline{Name: string}]\}

\item Has: \{[\underline{Title: string, DateOfRelease: datetime, Name: string $ \rightarrow $ User}, Language: string]\}
	Foreign key \{DateOfRelease, Title\} $ \rightarrow $ \{Movie.DateOfRelease, Movie.Title\}

\item Role: \{[\underline{RoleId: integer, DateOfRelease: datetime, Title: string, FirstName: string},\\ \underline{LastName:string, DateOfBirth: datetime}]\}

Foreign key \{DateOfRelease, Title\} $ \rightarrow $ \{Movie.DateOfRelease, Movie.Title\}

\{FirstName, LastName, DateOfBirth\} $ \rightarrow $ \{Person.FirstName, Person.LastName, Person.DateOfBirth\}

\item Director: \{[\underline{ID}]\}

\item Writer: \{[\underline{ID}]\}

\item Actor: \{[\underline{ID}, PartName]\}

\item Person: \{[\underline{FirstName: string, LastName: string, DateOfBirth: datetime,} Gender: boolean, Age:int]\}

\item Organization: \{[\underline{Name: string}]\}

\item Award: \{[\underline{Title: string}]\}

\item Nominated: \{[\underline{Type $ \rightarrow $ Award, Name $ \rightarrow $ Organization, DateOfRelease: datetime,}\\ \underline{Title: string, FirstName, LastName, DateOfBirth}, Year: int, Won: boolean,]\}

Foreign key \{Title,DateOfRelease\} $ \rightarrow $ \{Movie.Title, Movie.DateOfRelease\}
		\{FirstName, LastName, DateOfBirth\} $ \rightarrow $ \{Person.FirstName, Person.LastName, 
Person.DateOfBirth\}
\end{itemize}

\subsection{Keys}

Candidate keys for the database schema.
The selected primary keys are underlined.

\begin{center}
\begin{tabular}{| c | c |}
\hline
EntityType & Candidate keys\\
\hline 
\hline
Movie & \underline{DateOfRelease, Title}\\
\hline
Person & \underline{FirstName, LastName, DateOfBirth}\\
\hline
Role & \underline{ID}\\
\hline
\hline 
Award & \underline{Title}\\
\hline
Organization & \underline{Name}\\
\hline
User & \underline{Email}, Username\\
\hline
\end{tabular}
\end{center}

\section{Reflections}
Initially we didn't have a proper overview, since we only had a bunch of tables and a lot of explanatory text.
Now we have achieved this, by using an ER diagram.
By using an ER diagram, a much more formal notation, we have achieved a more precise and concise way of describing the database schema.
Additionally our initial attempt didn't show anything about cardinalities.
We used IDs for primary keys for all entity types, now that we have learned about keys, we have found a simpler and more elegant approach.
