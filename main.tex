\documentclass{report}
\renewcommand*\thesection{\arabic{section}}
\usepackage{amssymb}
\usepackage{amsmath}
\usepackage{amsthm}
\newtheorem{definition}{Definition}
\usepackage[linesnumbered, lined, noend]{algorithm2e}
\usepackage{todonotes}
%\usepackage[cm]{fullpage}
\usepackage{hyperref}
\hypersetup{
    colorlinks=true,       % false: boxed links; true: colored links
    linkcolor=black,       % color of internal links                   red
    citecolor=black,       % color of links to bibliography            green
    filecolor=black,       % color of file links                       magenta
    urlcolor=blue         % color of external links                   cyan
    }
\newcommand{\BigTheta}[1]{\ensuremath{\operatorname{\Theta}\bigl(#1\bigr)}}

\newcommand{\AND}{\wedge}
\newcommand{\OR}{\vee}
\newcommand{\stefan}[1]{\todo[inline,color=red]{#1}}
\newcommand{\mikkel}[1]{\todo[inline,color=yellow]{#1}}
\newcommand{\bruno}[1]{\todo[inline,color=blue]{#1}}
\newcommand{\mikael}[1]{\todo[inline,color=green]{#1}}

\usepackage{tikz-er2}

%Math database join symbols

\def\ojoin{\setbox0=\hbox{$\bowtie$}%
  \rule[-.02ex]{.25em}{.4pt}\llap{\rule[\ht0]{.25em}{.4pt}}}
\def\leftouterjoin{\mathbin{\ojoin\mkern-5.8mu\bowtie}}
\def\rightouterjoin{\mathbin{\bowtie\mkern-5.8mu\ojoin}}
\def\fullouterjoin{\mathbin{\ojoin\mkern-5.8mu\bowtie\mkern-5.8mu\ojoin}}
\newcommand{\HRule}{\rule{\linewidth}{0.5mm}}
\begin{document}

\begin{titlepage}
\centering
{\LARGE Database System - Mini-project}
\HRule \\[0.5cm]
Bruno Thalmann\\
			Mikael E. Christensen\\
			Mikkel S. Larsen\\
			Stefan M. G. Micheelsen
\end{titlepage}

\chapter*{Database Systems}

\section{Miniproject Part 1}
The following is a description of our database design of a movie database, at the beginning of the Database Systems course.
It is therefore our own attempt at designing the database.

The presentation in this section consists of the tables of database.
The tables are filled with some example entries to exemplify the usage of the database, as well as a textual description of the design.

\paragraph{General notation}
The notation \textbf{\textit{Data}Id} is used to refer to the id attribute in the \textit{Data} table.
That is, \textit{MovieID} in actors refers to the ID attribute in the \textit{Movies} table.
\stefan{jeg har beholdt dette afsnit, men det er måske lidt out of place her}

\subsection{Movies}
The table movies contains all the movies in the database. 
A movie entry consists of an id, a title and the year the movie was released.
Additional information about the movie is available by looking up the movie id in the tables ``actors'', ``directors'', ``writers'' and ``awards''

\paragraph{Movies}
\begin{center}
\begin{tabular}{|l|l|l|}
\hline
\multicolumn{3}{|c|}{Movies} \\ \hline \hline
\textbf{ID} & \textbf{Title} & \textbf{Year} \\ \hline \hline
1 & The Incredible Hulk & 2008 \\ \hline
2 & Fight Club & 1999 \\ \hline
\end{tabular}
\end{center}

\subsection{People}
All people associated with movies are stored in a single table.
This has the implication that a single \textit{Person} entity can be used as actor, writer and director for several movies.
The structure is chosen for its ability to facilitate lookups based on a single person or a single movie.
Associating people with a movie is done through the \textit{Actors}, \textit{Writers} and \textit{Directors} tables, indirectly declaring the role the person had in the production of the movie.

\paragraph{Persons}
Entries in the table below describes a single person.
An entry consists of a unique id for identification of a single person, the persons name (first- and surname), date of birth, nationality and sex.
Information about which movies a person has appeared in is available through lookup in the tables described above.

\begin{center}
\begin{tabular}{|l|l|l|l|l|l|}
\hline
\multicolumn{6}{|c|}{Persons} \\ \hline \hline
\textbf{ID} & \textbf{FirstName} & \textbf{SurName} & \textbf{Birthdate} & \textbf{Nationality} & \textbf{Sex} \\ \hline \hline
7 & Edward & Norton & 18-08-1969 & US & Male \\ \hline
9 & Brad & Pitt & 18-12-1963 & US & Male \\ \hline
10 & Helena & Bonham Carter & 26-05-1966 & UK & Female \\ \hline
13 & David & Fincher & 28-08-1962 & US & Male \\ \hline
14 & Chuck & Palahniuk & 21-02-1962 & US & Male \\ \hline
15 & Jim & Uhls & 25-03-1957 & US & Male \\ \hline
21 & Liv & Tyler & 01-07-1977 & US & Female \\ \hline
22 & Tim & Roth & 14-03-1961 & UK & Male \\ \hline
23 & Louis & Leterrier & 17-06-1973 & Fr & Male \\ \hline
24 & Zak & Penn & 00-00-1968 & N/A & Male \\ \hline
\end{tabular}
\end{center}

\paragraph{Actors}
Rows in the table below associates a person with a movie through an acting role in the movie.
To this end each row holds the name of the role played by the actor as well as the id of both a person (the actor) and a movie.
\begin{center}
\begin{tabular}{|l|l|l|l|}
\hline
\multicolumn{4}{|c|}{Actors} \\ \hline \hline
\textbf{ID} & \textbf{RoleName} & \textbf{PersonID} & \textbf{MovieID} \\ \hline \hline
1 & The Narrator & 7 & 1 \\ \hline
2 & Bruce Banner & 7 & 2 \\ \hline
3 & Tyler Durden & 9 & 2 \\ \hline
4 & Marla Singer & 10 & 2 \\ \hline
5 & Betty Ross & 21 & 1 \\ \hline
6 & Emil Blonsky & 22 & 2 \\\hline 
\end{tabular}
\end{center}

\paragraph{Directors}
Rows in the table below associates a person with a movie through the role of a movie director.
To this end each row holds the id of both a person (the director) and a movie.
\begin{center}
\begin{tabular}{|l|l|l|}
\hline
\multicolumn{3}{|c|}{Directors} \\ \hline \hline
\textbf{ID} & \textbf{PersonID} & \textbf{MovieID} \\ \hline \hline
1 & 23 & 2 \\ \hline
2 & 13 & 1 \\ \hline
\end{tabular}
\end{center}

\paragraph{Writers}
Rows in the table below associates a person with a movie through the role of a movie script writer.
To this end each row holds the id of both a person (the writer) and a movie.
\begin{center}
\begin{tabular}{|l|l|l|}
\hline
\multicolumn{3}{|c|}{Writers} \\ \hline \hline
\textbf{ID} & \textbf{PersonID} & \textbf{MovieID} \\ \hline \hline
1 & 14 & 2 \\ \hline
2 & 24 & 1 \\ \hline
1 & 15 & 2 \\ \hline
\end{tabular}
\end{center}

\subsection{Awards}

Awards consists of four tables, in order to handle linking an award to the appropriate type, organization and movie.

\paragraph{Awards} is the main table, linking all the data together.
It consists of the following five properties.

Like all other tables it contains an ID.

\textit{Year} is the appropriate year when the award was awarded.

\textit{AwardTypeID} links it to an \textit{AwardType}.
\textit{AwardOrganizationID} links it to an \textit{AwardOrganization}.
This is done to avoid inconsistencies when querying and to avoid redundant data.

The final property is \textit{MovieID}, linking it to the appropriate \textit{Movie}.

\begin{center}
\begin{tabular}{|l|l|l|l|l|}
\hline
\multicolumn{5}{|c|}{Awards} \\ \hline \hline
\textbf{ID} & \textbf{Year} & \textbf{AwardTypeID} & \textbf{AwardOrganizationID} & \textbf{MovieID} \\ \hline \hline
1 & 2000 & 1 & 1 & 2 \\ \hline
2 & 2001 & 2 & 2 & 2 \\ \hline
2 & 2001 & 3 & 2 & 2 \\ \hline
2 & 2001 & 4 & 2 & 2 \\ \hline
\end{tabular}
\end{center}

\paragraph{AwardWinners} is used to link a \textit{Person} to an \textit{Award}.
E.g. ''Helena Bonham Carter'' with \textit{PersonID} 10 won the \textit{Award} with \textit{AwardID} 1.

\begin{center}
\begin{tabular}{|l|l|l|}
\hline
\multicolumn{3}{|c|}{AwardWinners} \\ \hline \hline
\textbf{ID} & \textbf{AwardID} & \textbf{PersonID} \\ \hline \hline
1 & 1 & 10 \\ \hline
\end{tabular}
\end{center}

\paragraph{AwardTypes}

\begin{center}
\begin{tabular}{|l|l|}
\hline
\multicolumn{2}{|c|}{AwardTypes} \\ \hline \hline
\textbf{ID} & \textbf{Title} \\ \hline \hline
1 & Best British Actress \\ \hline
2 & Best DVD \\ \hline
3 & Best DVD Special Features \\ \hline
4 & Best DVD Commentary \\ \hline
\end{tabular}
\end{center}

\paragraph{AwardOrganizations}

\begin{center}
\begin{tabular}{|l|l|}
\hline
\multicolumn{2}{|c|}{AwardOrganizations} \\ \hline \hline
\textbf{ID} & \textbf{Name} \\ \hline \hline
1 & Empire Awards, UK \\ \hline
2 & Online Film Critics Society Awards \\ \hline
\end{tabular}
\end{center}

\subsection{Ratings}
The rating table contains all the ratings in the database and consists of an ID, Rating, a UserID and a MovieID.
The users table contains all the users in the database and stores the Username, Email and Password of each user.\\
Each rating is connected with a UserID and a MovieID.
Fx if we look at the rating with ID 3, the rating is \textit{61}.
The UserID attached to the rating is 1, this means we can look up that user in the users table.
The user with the ID 1 is \textit{Stufkan}.
The movie attached to the rating has ID 0.
If we look in the movies table the movie with ID 0 is \textit{The Incredible Hulk}.
\paragraph{Ratings}

\begin{center}
\begin{tabular}{|l|l|l|l|}
\hline
\multicolumn{4}{|c|}{Ratings} \\ \hline \hline
\textbf{ID} & \textbf{Rating} & \textbf{UserID} & \textbf{MovieID} \\ \hline \hline
1 & 80 & 0 & 1 \\ \hline
2 & 65 & 0 & 0 \\ \hline
3 & 32 & 1 & 1 \\ \hline
4 & 61 & 1 & 0 \\ \hline
5 & 12 & 2 & 1 \\ \hline
6 & 64 & 3 & 0 \\ \hline
7 & 99 & 4 & 1 \\ \hline
8 & 55 & 4 & 0 \\ \hline
\end{tabular}
\end{center}

\paragraph{Users}

\begin{center}
\begin{tabular}{|l|l|l|l|}
\hline
\multicolumn{4}{|c|}{Users} \\ \hline \hline
\textbf{ID} & \textbf{Username} & \textbf{Email} & \textbf{Password} \\ \hline \hline
1 & VisualStudioGuy & visualstudioguy@gmail.com & 123456 \\ \hline
2 & Stufkan & stufkan@gmail.com & 654321 \\ \hline
3 & GermanGuy & germanguy@gmail.com & bold1234 \\ \hline
4 & PizzaGuy & pizzaguy@gmail.com & pizza1234 \\ \hline
\end{tabular}
\end{center}

\end{document}